\documentclass[13pt,a4paper]{article}
\usepackage[utf8]{inputenc}
\usepackage[spanish]{babel}
\usepackage{makeidx}
\usepackage{graphicx}
\usepackage{latexsym,amsmath,amssymb,amsfonts} 
\usepackage[left=2cm,right=2cm,top=2cm,bottom=2cm]{geometry}


\title{\textbf{Práctica 02}}
\date{}
\author{Meneses Conde Jhon Saul}
\begin{document}

\maketitle
\begin{enumerate}
\item Dé un ejemplo de experimento aleatorio que es de interés para:un ingeniero electricista, un economista y un gerente de compañia de automoviles.
\begin{enumerate}a)
\item Un ingeniero electricista \\[0.1cm]
Solución:\\[0.1cm]
\begin{enumerate}a_1) Observar el tiempo de vida útil de un artefacto eléctrico.
\end{enumerate}\\[0.1cm]
\end{enumerate}
\begin{enumerate}b Un economista\\[0.1cm]
Solución:\\[0.1cm]
\begin{enumerate}b_1) Proyectarse la tasa de devaluation de la moneda.\\[0.1cm]
\end{enumerate}
\end{enumerate}
\begin{enumerate}c) Un gerente de una compañía de automóviles.\\[0.1cm]
Solución:\\[0.1cm]
\begin{enumerate}c_1)comprar por lo menos 10 vehículos blindados.\\[0.1cm]
\end{enumerate}\\[0.01cm]} 
\end{enumerate}

\item Construir El espacio muestral apropiado para los siguientes experimentos aleatorios.\\[0.1cm]
\begin{enumerate}a) Elegir una carta de una baraja de 52 cartas.\\[0.1cm] 
Solución:\\[0.1cm]
\begin{enumerate}a_1) \Omega ={Del 1 al 13 de diamantes(D), de corazones(C), de treboles(T) y de espadas(E).\\[0.1cm]
\end{enumerate}\\[0.1cm]\Rightarrow\Omega =\lbrace C_1,C_2,C_3,...,C_13,T_1,T_2,T_3,...,T_13,D_1,D_2,D_3,...,D_13,E_1,E_2,E_3,...,E_13. \rbrac
\end{enumerate}\\[0.1cm]

\begin{enumerate}b) Verificar el estado de dos transistores (apagado o encendido).\\[0.1cm]
solucion:\\[0.1cm]
\begin{enumerate}b_1) \Omega =\lbrace{ Encendido-Encendido, Encendido- Apagado, Apagado-Encendido, Apagado-Apagado}\rbrac
\end{enumerate}
\end{enumerate}\\[0.1cm]

\begin{enumerate}c)Verificar el estado de 10 transistores (apagado o encendido .\\[0.1cm] 
solucion:\\[0.1cm]
\begin{enumerate}c_1)\Omega ={\lbrace{ Encendido-Encendido, Encendido- Apagado, Apagado-Encendido,..., Apagado-Apagado}\rbrac}\\[0.1cm]
\ast En éste caso el espacio muestral tiene 100 posibles eventos, por lo que es muy dificil crearlo,\\[0.1cm]
 pero va a ser todas las posibles combinaciones entre encendidos y apagados.
\end{enumerate}
\end{enumerate}

\begin{enumerate}d) Se lanzan n monedas y se observa el número de caras.\\[0.1cm] 
solucion:\\[0.1cm]
\begin{enumerate}d_1)\Omega =\lbrace$$(x+a)^n\rbrace\\[0.1cm] 
\begin{center}
{ \fboxsep 12pt

\begin{minipage}[t]{10cm}

$$(x+a)^n=\sum_{k=0}^n \binom{n}{k}x^k a^{n-k}$$
\end{minipage}
} }
\end{center}
\end{enumerate}}
\end{enumerate}

\item  Un inversionista planea escoger dos de las cinco oportunidades de inversión que le han recomendado. Describa el espacio muestral que representa las opciones posibles.\\[0.1cm] 
solucion:\\[0.1cm]
\ast el espacio muestral de los cinco oportunidades de inversión.\\[0.1cm]
\Omega=\lbrace 1,2,3,4,5\rbrace\\[0.1cm]
\ast planea escoger dos de las cinco oportunidades de inversión.\\[0.1cm]
\Omega=\lbrace (1,1);(1,2);(1,3);(1,4);(1,5);(2,1);(2,2);(2,3);(2,4);(2,5);(3,1);(3,2);\\[0.1cm]
(3,3);(3,4);(3,5);(4,1);(4,2);(4,3);(4,4);(4,5);(5,1);(5,2);(5,3);(5,4);(5,5)\rbrace\\[0.1cm]

\item  Tres artículos son extraídos con reposición, de un lote de mercancías; cada artículo ha de ser identificado como defectuosos "D" y no defectuoso "N“. Describa todos los puntos posibles del espacio muestral para este experimento.\\[0.1cm] 
solucion:\\[0.1cm]
Los tres articulos son\lbrace 1,2,3\rbrace\\[0.1cm] 
D: Defectoso\\[0.1cm]
N:No defectoso\\[0.1cm]
\Omega=\lbrace(x,y)/x=1,2,3; y=D,N \rbrace\\[0.1cm] 
\Omega=\lbrace(1,D)(1,N);(2,D);(2,N);(3,D);(3,N) \rbrace\\[0.1cm]

\item Dos personas A y B se distribuyen al azar en tres oficinas numerada 1, 2 y 3. Si las dos personas pueden estar en la misma oficina, defina un espa_ ció muestral adecuado.\\[0.1cm] 
solucion:\\[0.1cm]
A: Persona 1\\[0.1cm]
B:Persona 2\\[0.1cm] 
Numero de oficinas  son\lbrace 1,2,3\rbrace\\[0.1cm] 
\Omega=\lbrace(x,y)/x=A,B; y=1,2,3 \rbrace\\[0.1cm] 
\Omega=\lbrace(A,1)(B,1);(A,2);(B,2);(A,3);(B,3) \rbrace\\[0.1cm]


\item  Tres personas A , B y C se distribuyen al azar en dos oficinas numeradas con 1 y 2. Describa un espacio muestral adecuado a este experimento.
\begin{enumerate}a) si los tres pueden estar en una misma oficina\\[0.1cm]
solucion:\\[0.1cm]
A: Persona 1\\[0.1cm]
B:Persona 2\\[0.1cm] 
C:Persona 3\\[0.1cm] 
Numero de oficinas  son\lbrace 1,2\rbrace\\[0.1cm] 
\Omega=\lbrace(x,y)/x=A,B,C ; y=1,2 \rbrace\\[0.1cm] 
\Omega=\lbrace(A,1)(B,1);(C,1);(A,2);(B,2);(B,3) \rbrace\\[0.1cm] 

\end{enumerate}
\begin{enumerate}B)sí sólo se puede asignar una persona a cada oficina.\\[0.1cm] 
solucion:\\[0.1cm] 
\Omega=\lbrace((A,1),(B,1));((C,1),(A,2));((B,2),(B,3));((A,1),(B,3));((B,1),(B,2));((A,2),(C,1)) \rbrace\\[0.1cm] 
\end{enumerate}

\item Durante el día, una máquina produce tres artículos cuya calidad individual, definida como defectuoso o no defectuoso, se determina al final del día. Describa el espacio muestral generado por la producción diaria.\\[0.1cm]
solucion:\\[0.1cm]
\Omega =X_1,X_2,X_3\\[0.1cm]
X_i=D,B ;      i=1,2,3\\[0.1cm]
donde:\\[0.1cm]
D:Defectuoso\\[0.1cm]
B:No defectoso\\[0.1cm]
\Omega=\lbrace (X_1D,X_2B,X_3B);(X_1B,X_2D,X_3B);(X_1B,X_2B,X_3D);\\[0.1cm]
(X_1D,X_2D,X_3D);(X_1D,X_2D,X_3B);(X_1D,X_2B,X_3D);(X_1B,X_2D,X_3D);(X_1B,X_2B,X_3B)\rbrace\\[0.1cm] 
\Omega=\lbrace DDD,DDB,DBD,BDD,BBD,BDB,DBB,BBB\rbrace\\[0.1cm] 

\item  El ala de un avión se ensambla con un número grande de remaches. Se inspecciona una sola unidad y el factor de importancia es el número de remaches defectuosos. Describa el espacio muestral.\\[0.1cm]
solucion:\\[0.1cm]

El número de remaches de un avión es un gran número que podemos considerar infinito.

X=nº remaches defectuosos  tiene un espacio muestral 

\Omega =\lbrace{0,1,2,3,4,..... }\rbrace = N \cup \lbrace{0}\rbrace

\item  Suponga que la demanda diaria de gasolina en una estación de servicio está acotada por 1,000 galones, que se lleva a un registro diario de venta. Describa el espacio muestral.\\[0.1cm]
solucion:\\[0.1cm] 

\item Se desea medir la resistencia al corte de dos puntos de soldadura. Suponiendo que el límite superior está dado por U, describa el espacio muestral .\\[0.1cm]
solucion:\\[0.1cm] 

\item  De un grupo de transistores producidos bajo condiciones similares, se eScoge una sola unidad, se coloca bajo prueba en un ambiente similar a su uso diseñado y luego se prueba hasta que falla. Describir el espacio muestral. \\[0.1cm]
solucion:\\[0.1cm]


\end{enumerate}
\end{document}